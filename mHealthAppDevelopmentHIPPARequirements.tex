
\documentclass{article}
\usepackage[utf8]{inputenc}
\begin{document}
\begin{center}
\Huge mHealth App Development HIPPA Requirements
\end{center}
\section*{What is mHealth}
mHealth is mobile health. It is used to refer to consumer health apps by health care professionals. It includes wearables, but doesn't have to.

\section*{What are mHealth Applications?}
Any software that runs on smartphones or tablets and manages or tracks personal health. This can include physical activity or biometrics such as health rate and prescription regimens. Some examples are RunKeeper and Couch to 5k. Sleep monitoring apps are also considered mHealth, as are heart rate monitors. 

\section*{HIPPA Compliance and mHealth Applications}
HIPPA stands for Health Insurance Portability and Accountability Act. This is a standard for sensitive patient data. It was created in 1996 and now it applies to a lot of technology. App developers have to be concerned with this if their app accesses 'medical records, billing information, health insurance information, and any individually identifiable health information.' This is considered private health information, or PHI. PHI is not 'calories burned, steps taken, or distance covered.' That means things like Nike Fuelband do not fall under accessing PHI, whereas things like Apple's health app do. If an app does track or access PHI then the app has to comply with HIPPA. 

\section*{Developing HIPPA Compliant mHealth Applications}
Apparently using True Vault 'will ensure that you meet the technical and physical safegaurds required by the HIPPA Security Rule.' Will have to look more into this later, considering this is coming from the website itself. 

\end{document}
