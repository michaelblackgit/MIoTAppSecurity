
\documentclass{article}
\usepackage[utf8]{inputenc}
\begin{document}
\begin{center}
\Huge Privacy and Security in Mobile Health (mHealth) Research
\end{center}
\section*{Introduction}
New advancements in technology allow scientists to collect info from patients using real time wearables. These can use sensors and can provide a patient's 'biology, psychology (attitudes, cognitions, and emotions), behavior and daily environment.' This can give new insights into diseases and preventative measures for diseases and optimize a patient's outcomes and possibilities of recovery due to mHealth devices' ability to provide a constant stream of data. Even though mHealth has great potential it has progressed slowly due to concerns for privacy and security. Great quote: 'Because most mobile devices (including phones and sensors) are carried by the person and collecting data throughout the day, researchers are now able to begin thinking about big data at the level of the individual (Estrin 2014).' Followed by another great quote: 'Fusion of streaming biological, physiological, social, behavioral, environmental, and locational data can now dwarf the traditional genetics and electronic health records-based datasets of so-called big data.' People of who recently could not participate in research now can due to the accessibility of smartphones. When dealing with research participants, observers and health care professionals must be aware of the interconnected web of privacy, security, and confidentiality concerns that mHealth apps present. The National Committee for Vital and Health Statistics describes the differences between privacy, security and confidentiality with the following: 'Health information privacy is an individual’s right to control the acquisition, uses, or disclosures of his or her identifiable health data. Confidentiality, which is closely related, refers to the obligations of those who receive information to respect the privacy interests of those to whom the data relate. Security is altogether different. It refers to physical, technological, or administrative safeguards or tools used to protect identifiable health data from unwarranted access or disclosure (Cohn 2006).' This article chooses to point out security issues regarding the specific mHealth applications of alcohol, drug use, and mental health. 

\end{document}
