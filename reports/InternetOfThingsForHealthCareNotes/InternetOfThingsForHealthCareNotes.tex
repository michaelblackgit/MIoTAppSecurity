\documentclass{article}
\usepackage[utf8]{inputenc}
\begin{document}
\begin{center}
\LARGE The Internet Of Things For Health Care Notes
\end{center}
\section*{The IoThNet Topology}
Transforms heterogeneous computing and storage capability of mobile devices into hybrid 
computing grids.
Heterogenous computing: System that uses more than one kind of processor.
Hybrid computing grids: Cloud.
Basically the devices and the data providers, i.e. sensors, talk to each other, getting 
and posting information under the surveilance of a Broker in a central space called the
Coordination Space.

\section*{The IoTheNet Architecture}
The physical elements. Continua Health Aliance recomends the following: An IoT gateway and
wireless local network (WLAN) or a personal area network (WPAN) along with multimedia 
streaming and secure communications between these. According to this article, the 
IPv6-based 6LoWPAN is the basis of the IoThNet. This is used by sensors and wearables
to transmit data. But this doesn't support mobile devices. These instead need HTTP, COAP,
and SSL for the application. The transport uses TCP or UDP (WebSockets?) and the network
is either IPv6 or RPL. The adaptation is 6LoWPAN Adaptation and the Link and PHY is
IEEE 802.15.4 PHY/MAC.
To make transfer of data fast for these applications they use a DAG.

\section*{The IoThNet Platform}
Refers to both the network and computing platform. This may be where middlewares like
KAA come in. There is a design for interoperability of hardware and software that 
includes human-machine interfaces, multidisciplinary optimization, and application
management. This is an interface standardization. 
In one case there is the application of using VITRUS as a middleware, based on 
an instand messaging protocol, XMPP. Multiple users with multiple sensors need to be 
able to acess the gateway even with poor connectivity and to do this an algorithm is 
used to read raw health data from an edge router and parse its data in a predetermined
format. A three-layer cloud platform is used to access cloud data and includes a resource
layer for controlling the access to data and a business layer for the coordination of data
sharing.

\section*{IoT Healthcare Services and Applications}
This paper splits apart the offer of IoT-based healthcare systems into services and 
applications. Applications themselves are divided into two groups: single-condition and
cluster-condition applications. The first is for a specific disease while the latter
is many diseases treated as a whole. 

\section*{IoT Healthcare Services}
The point of services is to provide healthcare solutions. Thus a service is generically
a setup for some kind of solution. Solutions can include notifications, resource-sharing
devices, internet services, cross-connectivity protocols for heterogeneous devices, and
link protocols for major connectivity. It may also aim to make the discovery of devices
quick and secure. 

\begin{enumerate}
\item Ambient Assisted Living (AAL)
\\ This is artificial intelligence that is used to take care of the elderly or those who 
are incapacitated. Its primary focus is to make these people more independent and 
increase the quality of their life. For AAL, radio frequency identification (RFID) and
near-field communications (NFC) are used for passive communication. Keep in touch (KIT)
and closed-loop health services are said to help facilitate AAL. 

\item The Internet of m-Health Things (m-IoT)
\\ This is mobile computing, medical sensors, and communication tech for healthcare services. 
It represents all of IoT for healthcare services, but has some attributes that are unique
to it. It has the potential to of m-IoT for the noninvasive sensing of glucose levels. 
Two challenges in this are context-aware issues and m-IoT ecosystems. Some of these services
are based on mobile message exchange.

\item Adverse Drug Reaction (ADR)
\\ This is an injury from taking a medication. This is very generic, thus ADR services vary
depending on the drug at hand. An example of this is using a barcode through NFC to 
track whether the drug a patient is using is compatible with logged alergies and other
tracked health records.

\item Community Healthcare (CH)
\\ This involves establishing a network around a local community, like a hospital or a rural
community, through a cooperative network structure, and it treats many health requirements
as a package. This can be viewed as a virtual hospital. 

\item Children Health Information (CHI)
\\ This can be applied to emotional, behavioral, or mental health. This can also be 
entertaining for children.

\item Wearable Device Access (WDA)
\\ This utilizes nointrusive sensors, smartphones, and smart watches. Bluetooth may also be
a prospect.

\item Semantic Medical Access (SMA)
\\ This is basically used for the analysis of large amounts of data, such as ontologies.

\item Indirect Emergency Healthcare (IEH)
\\ This is used for emergencies and can provide information, notifications, post-accident
action, and record keeping.

\item Embedded Gateway Configuration (EGC)
\\ This connects patients to the internet and other medical equipment. It should allow for
automated and intelligent monitoring. This could include mobile computing devices. 
\end{enumerate}
\end{document}

